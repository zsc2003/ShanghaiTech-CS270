\problem{}
For the details of this problem: the image is padded with the edge values during the convolution, since during the convolution process, it may need to access the pixels outside the image.

(a) The Laplacian kernel is that:
$$
\nabla^2 f = \begin{bmatrix}
    0 & 1 & 0 \\
    1 & -4 & 1 \\
    0 & 1 & 0
\end{bmatrix}
=
\begin{bmatrix}
    0 & 1 & 0 \\
    0 & -2 & 0 \\
    0 & 1 & 0
\end{bmatrix}
+
\begin{bmatrix}
    0 & 0 & 0 \\
    1 & -2 & 1 \\
    0 & 0 & 0
\end{bmatrix}
$$
We can seperate the Laplcian kernel along the $x$-direction and $y$-direction, and we can simplify them into 1-D:

$$\begin{bmatrix}1\\-2\\1\end{bmatrix} \text{\ \ and \ \ } \begin{bmatrix}1 & -2 & 1\end{bmatrix}$$

The processed image corresponding to the kernel above all shown in Figure \ref{fig:p2a}.\\

\begin{figure}[htbp]
    \centering
	\includegraphics[width=1\textwidth]{../images/p2/p2a.png}
    \caption{Separated Laplacian kernels processed image}
\label{fig:p2a}
\end{figure}


(b) The Sharpened image with Laplacian kernel is shown in Figure \ref{fig:p2b}.\\
\begin{figure}[htbp]
    \centering
    \includegraphics[width=\textwidth]{../images/p2/p2b.png}
    \caption{Unseparated Laplacian kernel processed image}
\label{fig:p2b}
\end{figure}

The image sharpening with Laplacian could be seen as sharpened with the kernel
$$
\begin{bmatrix}
    0 & -1 & 0 \\
    -1 & 5 & -1 \\
    0 & -1 & 0
\end{bmatrix}
$$

In order to make the right side of the image to be black instead of being gray, the $<0$ and $>255$ part are abandoned, i.e. turning the $<0$ part into $0$, and turning the $>255$ part into $255$,
which is shown in the left image.\\
And the right image is doing no additional operations, slighting mapping $[\text{min},\text{max}]\to[0,255]$.\\

(c) The Sharpened image with unsharpen mask is shown in Figure \ref{fig:p2c}.\\
The first two rows (marked filter1 in the title) are smoothed with the kernel 
$$\text{kernel\ 1}=\dfrac{1}{9}\begin{bmatrix}1 & 1 & 1\\1 & 1 & 1\\1 & 1 & 1\end{bmatrix}$$

And the last two rows (marked filter2 in the title) are smoothed with the kernel
$$\text{kernel\ 2}=\dfrac{1}{16}\begin{bmatrix}1 & 2 & 1\\2 & 4 & 2\\1 & 2 & 1\end{bmatrix}$$

Suppose the origin image is $f(x,y)$.\\
The first column are the smoothed images processed by kernel1 and kernel2, mark as $\overline{f(x,y)}$.\\
The second column are the unsharped masks processed by kernel1 and kernel2. i.e. the difference between the
origin image and the smoothed image. i.e. $g_{mask}(x,y)=f(x,y)-\overline{f(x,y)}$.\\
The third and the forth column are the sharpened images with different $k$. i.e. $g(x,y)=f(x,y)+k\cdot g_{mask}(x,y)$\\
The third column is the sharpened image with $k=1$, and the forth column is the sharpened image with $k=4.5$.\\
And the fifth column is the origin image.\\

For the difference bewteen the $1,2$ rows and $3,4$ rows is that: the $1,3$ rows are doing normalization with $[\text{min},\text{max}]\to[0,255]$,
and the $2,4$ rows are setting $<0$ and $>255$ part are abandoned, i.e. turning the $<0$ part into $0$, and turning the $>255$ part into $255$.\\
We could see that as $k$ grows, more high frequency of the image increases, make the image sharper.\\

\begin{figure}[htbp]
    \centering
	\includegraphics[width=\textwidth]{../images/p2/p2c_1_no_drop.png}
	\includegraphics[width=\textwidth]{../images/p2/p2c_1_drop.png}
    \includegraphics[width=\textwidth]{../images/p2/p2c_2_no_drop.png}
	\includegraphics[width=\textwidth]{../images/p2/p2c_2_drop.png}
    \caption{unsharpen mask processed image}
\label{fig:p2c}
\end{figure}

\newpage