\problem{}
(a) Figure \ref{fig:p1a} is the histogram image of grain.tif.\\

\begin{figure}[htbp]
    \centering
	\includegraphics[width=\textwidth]{../images/p1/p1a.png}
    \caption{Histogram of grain.tif}
    \label{fig:p1a}
\end{figure}

(b) The left image of Figure \ref{fig:p1b} is the histogram equalized image, and the right one is 
the histogram of that histogram equalized image.\\

\begin{figure}[htbp]
    \centering
	\includegraphics[width=\textwidth]{../images/p1/p1b.png}
    \caption{Histogram equalized image and its histogram}
    \label{fig:p1b}
\end{figure}

(c) The left image of Figure \ref{fig:p1c} is the CLAHE processed image, and the right one is 
the histogram of that CLAHE processed image.\\

And for the details of CLAHE, the image is padded with the edge values, since during the CLAHE process, it may need to access the pixels outside the image.

\begin{figure}[htbp]
    \centering
	\includegraphics[width=\textwidth]{../images/p1/p1c.png}
    \caption{CLAHE processed image and its histogram}
    \label{fig:p1c}
\end{figure}

\newpage